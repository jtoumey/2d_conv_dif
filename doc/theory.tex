\documentclass{article}

\usepackage[utf8]{inputenc}
\usepackage{amsmath}


% Custom command definitions
\newcommand{\DA}[1]{D_{#1} A_{#1}}
\newcommand{\FA}[1]{F_{#1} A_{#1}}

\newcommand{\dby}[2]{\frac{\partial #1}{\partial #2}}
\newcommand{\intb}[1]{\int\limits_{#1}}

\newcommand{\lp}{\left(}
\newcommand{\rp}{\right)}


% Vector Calculus
\newcommand{\divc}[1]{\mathrm{div}\!\left(#1\right)}
\newcommand{\gradc}[1]{\mathrm{grad}\!\left(#1\right)}

\title{Two-Dimensional Convection-Diffusion}
\author{J.\ Toumey}
\date{Fall 2016}


\begin{document}
\maketitle


\section{Governing Equations}
The first equation governing the behavior of fluid in the domain is the two-dimensional convection-diffusion equation.

\begin{equation}
   \divc{\rho \mathbf{u} \phi} = \divc{\Gamma \, \mathrm{grad}\, \phi} + S_{\phi}
   \label{eqn:c-d}
\end{equation}

In the above equation, $\rho$ is the fluid density, $\mathbf{u}$ is the velocity vector, $\phi$ is the scalar which we transport, $\Gamma$ is the diffusion coefficient, and $S_{\phi}$ is the source term.

Note that the current implementation assumes that the flow is source-free so we will eliminate the $S_{\phi}$ term in later representations of this equation.


The second governing equation is the continuity equation (conservation of momentum).

\begin{equation}
   \divc{\rho \mathbf{u}} = 0
\end{equation}
\begin{equation}
   \dby{\rho u}{x} + \dby{\rho v}{y} = 0
\end{equation}

\section{Derivation of FVM Coefficents}
The discretization of the governing equation follows the method of Versteeg \& Malalasekera. First, we expand the convection-diffusion equation (\ref{eqn:c-d}) for two dimensions.

\begin{equation}
   \dby{\rho u \phi}{x} + \dby{\rho v \phi}{y} = \dby{}{x} \lp \Gamma \dby{\phi}{x} \rp + \dby{}{y} \lp \Gamma \dby{\phi}{y} \rp + S
\end{equation}

Next, integrate the governing equation over a discrete control volume.

\begin{equation}
   \intb{CV} \dby{\rho u \phi}{x}\,dV + \intb{CV} \dby{\rho v \phi}{y}\,dV =\intb{CV} \dby{}{x} \lp \Gamma \dby{\phi}{x} \rp\,dV + \intb{CV} \dby{}{y} \lp \Gamma \dby{\phi}{y} \rp\,dV + \intb{CV} S\,dV
\end{equation}





In the equation below, $\mathrm{nb}$ indicates a neighbor node. 

\begin{equation}
   a_P \phi_P = \sum a_{\mathrm{nb}} \phi_{\mathrm{nb}} + S_u
\end{equation}

For the central coefficient $a_P$, we use equation \ref{eqn:ap} to determine the value after calculating all other coefficients.

\begin{equation}
   a_P = \sum a_{\mathrm{nb}} + \lp F_e - F_w \rp + \lp F_n - F_s \rp - S_P
   \label{eqn:ap}
\end{equation}

For interior cells, we can calculate the coefficients via the following formulations.
\begin{table}[!ht]
\centering
\begin{tabular}{|l|c|} \hline
   $a_W$ & $\DA{w} + max\left[0,  \FA{w}\right]$ \\ \hline
   $a_E$ & $\DA{e} + max\left[0, -\FA{e}\right]$ \\ \hline
   $a_S$ & $\DA{s} + max\left[0,  \FA{s}\right]$ \\ \hline
   $a_N$ & $D_n A_n + max\left[0, -F_n A_n\right]$ \\ \hline
   $S_u$ & 0 \\ \hline
   $S_P$ & 0 \\ \hline
\end{tabular}
\end{table}


\section{Algorithm Details}

\subsection{Pressure-Velocity Coupling}
For this implementation, we assume that the velocity field is known \textit{a priori}. Therefore, we store velocities, density, and the advected scalar $\phi$ at the cell center and avoid specific treatments of pressure-velocity coupling. These treatments include a staggered grid storage arrangement and predictor-corrector methods for iteratively determining the velocity field.


\end{document}
